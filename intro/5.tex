\hypertarget{0.5}{}
\head{Образовательные ресурсы}
В данном разделе приведён список полезных сайтов, на которых можно получить дополнительные знания по C++ или же порешать олимпиадные задачи. Часть информации в данной книге бралась из этих источников.

\begin{itemize}
    \item \href{https://stepik.org/course/363/}{Введение в программирование (C++)} — курс на Stepik от Яндекса. Не сложный, хороший для старта в C++.
    \item \href{https://stepik.org/course/7/}{Программирование на языке C++} — другой курс на Stepik от Computer Science Center. Сложнее курса от Яндекса.
    \item \href{https://stepik.org/course/3206/}{Программирование на языке C++ (продолжение)} — продолжения предыдущего курса. Он правда сложный, лучше сначала пройти первую часть курса :)
    \item \href{https://edu.sirius.online/}{Сириус Курсы} — на этом сайте ОЦ Сириус периодически выкладывает курсы по C++. Также можно подать заявку на программу Сириуса (по информатике), так как у некоторых есть вводные курсы по олимпиадному программированию :)
    \item \href{https://acmp.ru/}{Acmp} — сайт с задачами, создан в Красноярском крае.
    \item \href{https://informatics.msk.ru/}{Informatics} — тоже сайт с задачами.
    \item \href{https://codeforces.com/}{Codeforces} — ещё один сайт с задачами. но чуть популярней, есть пользователи со всего мира.
    \item \href{https://sort-me.org/}{SortMe} — относительно новый и перспективный сайт с задачами.
    \item \href{https://e-maxx.ru/algo/}{E-maxx} — сайт с алгоритмами.
    \item \href{https://ru.cppreference.com/w/}{Cppreference} и \href{https://cplusplus.com/reference/}{Cplusplus} — сайты с документацией C++. Стоит научиться читать документацию на английском, благо, что там используется очень ограниченный набор слов.
    \item «Антти Лааксонен. Олимпиадное программирование (2018)» — большая хорошая книжка, в которой есть как простые алгоритмы, так и сложные. Скачать можно \href{https://drive.google.com/file/d/13RWpaWUpPlMzVGGMzvMm6cjCxWzYzolI/view?usp=sharing}{отсюда}.
    \item «Стивен Халим, Феликс Халим. Спортивное программирование (2020)» — эта книга ещё больше, вроде бы в ней содержится достаточно много алгоритмов, но автор изучить её ещё не успел. Скачать книгу можно \href{https://drive.google.com/file/d/1TS8rambgTvplQ1Xpxl_GTbGZ3O9liV9C/view?usp=sharing}{отсюда}.
\end{itemize}
