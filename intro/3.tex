\head{Математика}
В олимпиадном программировании встречается довольно специфическая математика, которую может не знать читатель. Поэтому некоторые сведения из математика стоит привести заранее.

\subhead{Комбинаторика}
Факториалом натурального числа $n$ называется произведение чисел от одного до $n$:
$$n! = 1 \cdot 2 \cdot \ldots \cdot n = \prod\limits_{x = 1}^{n} x$$

При этом дополнительно определяется, что $0! = 1$.

Число перестановок
\footnote{Немного определений:

Перестановкой длины $n$ называется список из $n$ натуральных чисел от $1$ до $n$ (включительно). Каждое число должно встречаться ровно один раз.

Перестановка $a$ \term{лексикографически} меньше перестановки $b$, если первые $k$ элементов у них совпадают, а $k + 1$-ый меньше у $a$: $a_i = b_i$ и $a_{k + 1} < b_{k + 1}$, для $k \geq 0$ и $1 \leq i \leq k$.
}
$n$ элементов обозначается как $P_n$ и вычисляется как $P_n = n!$ (потому что первым может быть любой из $n$ элементов, вторым — любой из $n - 1$ оставшихся, и т.д.).

Число размещений $n$ элементов на $k$ позиций (порядок важен) обозначается как $A_n^k$ и вычисляется как $A_n^k = \frac{n!}{(n - k)!}$ (потому что первым может быть любой из $n$ элементов, вторым — любой из $n - 1$ оставшихся, и т.д., а последним — любой из $n - k + 1$, потому что потом $k$ мест уже закончатся).

Число сочетаний обозначается как $C_n^k$ и означает количество наборов размера $k$ из $n$ элементов, при этом порядок не важен. Вычисляется же оно как: $C_n^k = \frac{n!}{k!(n - k)!}$

Также у чисел сочетаний есть несколько свойств
\begin{enumerate}
    \item $C_n^k = C_n^{n - k}$ — потому что выбрать $k$ элементов это то же самое, что и не выбрать $n - k$ элементов.
    \item $C_n^k = C_{n - 1}^{k} + C_{n - 1}^{k - 1}$ — потому что $n$-ый элемент может быть либо выбран, либо не выбран. При этом стоит считать $C_n^n = C_n^0 = 1$.
    \item $C_n^0 + C_n^1 + \ldots + C_n^n = 2^n$ — потому что для каждого из $n$ элементов писать $0$, если элемент не взят и $1$ — если взят, также можно установить и обратную операцию. Так мы получаем однозначное соответствие между набором и двоичным числом, а значит формула верна.
    \item Бином Ньютона: $(a + b)^n = C_n^0\cdot a^n b^0 + C_n^1\cdot a^{n - 1} b^1 + \ldots + C_n^n\cdot a^0 b^n = \sum\limits_{k=0}^{n} C_n^k\cdot a^{n - k} b^k$ — потому что число сочетаний $C_n^k$ показывает, сколько есть способов выбрать множитель $b$ в ровно $k$ скобках. Отсюда второе название чисел сочетания — \term{биноминальные коэффициенты}.
\end{enumerate}

\subhead{Множества}
Множеством называется набор элементов, в котором не должно быть повторов. Над множествами можно совершать следующие операции (чем выше операция, тем выше её приоритет):

\starttable
\begin{tabular}{|c|c|c|}
\hline
Название & Запись & Описание \\
\hline
Мощность & $|A|$ или $\sharp A$ & Количество элементов в множестве. \\
Дополнение & $\overline{A}$ & Те элементы, которых нет в $A$. \\
Пересечение & $A \cap B$ & Те элементы, которые есть в обоих множествах. \\
Объединение & $A \cup B$ & Те элементы, которые есть хотя бы в одном множестве. \\
Разность & $A \setminus B$ & Те элементы, которые есть в $A$, но нет в $B$. \\
Симметрическая разность & $A \bigtriangleup B$ & Те элементы, которые есть ровно в одном множестве. \\
\hline
\end{tabular}
\endtable


\subhead{Логарифмы}
Логарифм числа $x$ по основанию $a$ обозначается, как $\log_a(x)$. Значение логарифма определяется следующим образом: если $b = \log_a(x)$, то $x = a^b$. Для логарифмов по основанию $e \approx 2.71828$ придумано специальное обозначение $\ln(x)$, и такой логарифм называется натуральным. Также есть десятичный логарифм (у него основание — 10), он обозначается, как: $\lg(x)$.

Например, поскольку $2^3 = 8$ и $\sqrt[3]{8} = 2$, то $\log_2(8) = 3$ (отсюда понятна необходимость введения логарифма, ведь у нас есть три переменных, значит нужно уметь выражать каждую через две оставшихся).

Для информатики полезно понимать, что логарифм $\log_a(x)$ показывает, сколько раз нужно число $x$ разделить на $a$, чтобы получить $1$. Например, для $32$ понадобится $5$ делений:
$$32 \to 16 \to 8 \to 4 \to 2 \to 1$$

Также можно привести некоторые свойства логарифмов:
\begin{enumerate}
    \item Логарифм произведения: $\log_a(xy) = \log_a(x) + \log_a(y)$
    \item Логарифм частного: $\log_a\left(\frac{x}{y}\right) = \log_a(x) - \log_a(y)$
    \item Логарифм степени (следствие произведения): $\log_a(x^y) = y\log_a(x)$
    \item Замена основания логарифма: $\log_a(x) \cdot \log_b(a) = \log_b(x)$, или $\log_a(x) = \frac{ \log_b(x) }{ \log_b(a) }$
\end{enumerate}
