\head{Порядок чтения}
И в последнем разделе вступительной части хотелось бы привести описание сложности и полезности всех последующих частей книги.

\begin{longtable}{|m{0.18\textwidth}|m{0.12\textwidth}|m{0.12\textwidth}|m{0.5\textwidth}|}
\hline
    \begin{center} Название \end{center} &
    \begin{center} Сложность понимания \end{center} &
    \begin{center} Сложность написания \end{center} &
    \begin{center} Полезность \end{center} \\
\hline
\endhead
    Введение в C++ & Просто & Просто & Стоит прочитать, если сомневаетесь, что знаете всё необходимое. Или можно возвращаться в процессе чтения. \\
\hline
    Идеи проектирования алгоритмов & Просто & Средне & Там собраны полезные идеи на многие случаи жизни -- прочитать стоит и применяется часто. \\
\hline
    Алгебра & Просто & Просто & Эти алгоритмы не встречаются как самостоятельные задачи, но могут быть подзадачами в более сложных. Стоит именно первой темой, чтобы попрактиковаться в написании алгоритмов, описанных словами. \\
\hline
    Сортировки & Просто & Просто и средне & Это классическая тема, которую рассказывают многим начинающим олимпиадникам. Сами по себе сортировки не нужно писать, так как уже есть встроенные, но они тоже учат писать описанный текстом алгоритм. \\
\hline
    Перебор & Просто & Просто и средне & Полезная тема, переборы каких-то объектов даже встречаются на олимпиадах. \\
\hline
    Жадные алгоритмы & Просто & Просто & Применимость жадных алгоритмов нужно чувствовать, а в главе приведены известные жадные алгоритмы. Тема встречается. \\
\hline
    Динамическое программирование & Просто & Просто и средне & Вот динамика встречается очень часто и в ней следует усиленно практиковаться. Приведены какие-то известные задачи на динамику. \\
\hline
    Бинарный поиск & Просто & Просто & Сам алгоритм очень короткий и применяется как подзадача в больших задачах. Читать нужно. \\
\hline
    Простые линейные алгоритмы & Просто & Просто & Раздел не самый полезный, и не сложный для понимания. Но мыслить в терминах линейных алгоритмов и понимать их иногда бывает полезно. \\
\hline
    Запросы суммы и минимума & Средне & Средне & Это уже более сложные линейные алгоритмы, и очень мало задач, в которых пригождаются. \\
\hline
    Сложные линейные алгоритмы & Средне & Средне & Этот раздел основан на предыдущем и тоже не отличается большой применимостью в задачах. \\
\hline
    Структуры данных & Средне и сложно & Средне и сложно & Это полезный раздел, но некоторые структуры сложны для понимания. Структуры в <<хорошем>> порядке: DSU, sqrt-декомпозиция, дерево отрезков, декартово дерево, разреженная таблица, дерево Фенвика, куча. Самая сложная структура из них -- ДД. \\
\hline
    Введение в графы & Просто & Просто & В этом разделе собраны разные термины про графы и то, как их хранить в памяти. Тема полезная, графы встречаются часто. \\
\hline
    Обходы графов & Среднее & Средне & Этот раздел тоже нужно читать, иначе с графами ничего сделать не сможете. Обходы нужны для дальнейшего понимания графов. \\
\hline
    Остовные деревья и кратчайшие пути & Средне и сложно & Средне и сложно & Здесь уже собраны какие-то более серьёзные алгоритмы, но они всё ещё могут быть как подзадачи. Но и кратчайшие пути сами по себе очень полезны. \\
\hline
    Геометрия & Средне и сложно & Средне и сложно & Общеизвестно, что мало людей любит геометрию. Но всё же задачи на неё бывают, поэтому прочитать стоит. В первой части много математики и геометрических примитивов, а вот во второй уже какие-то более содержательные алгоритмы. \\
\hline
    Алгоритмы на строках & Средне и сложно & Просто и средне & Чаще всего от строк нужно уметь считать хеши, это довольно не сложно. Но вот с z и префикс функциями могут возникнуть сложности с пониманием (и как следствие с написание), а вот бор довольно не сложен для понимания. \\
\hline
    Рандом & Просто и средне & Просто и средне & Короткая глава, с какими-то интересными штуками. В олимпиадах рандом обычно не встречается, так что глава больше для общего развития. \\
\hline
\end{longtable}

В целом, изучать книгу можно в достаточно произвольном порядке, но если идти последовательно, то алгоритмы будут последовательно зависеть друг от друга (то есть текущей алгоритм не будет опираться на ещё не изученные). Вполне может быть, что порядок тем стоит поменять -- можете предлагать это автору.
