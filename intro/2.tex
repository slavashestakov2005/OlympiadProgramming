\head{Установка Code::Blocks}
Своё знакомство с C++ мы начнём с установки среды разработки. Вы можете задать логичный вопрос: чем так плохи online-компиляторы? Ответ простой: на почти всех олимпиадах доступ в интернет запрещён, а значит, можно использовать только программы, установленные на компьютере. Вторая причина — online компиляторы, в среднем, работают медленнее установленных на компьютер. И, наконец, третья причина, не столь олимпиадная — если вы хотите создавать на C++ большие программы, то их придётся разбивать на много файлов, а под это online-компиляторы не подходят.

Теперь можем перейти непосредственно к установке среды разработки. Для C++ их, конечно, существует много разных, но здесь будет описан процесс установки Code::Blocks, так как это бесплатная IDE с открытым исходным кодом, к тому же, сама среда минималистична и занимает мало места в памяти компьютера.

Но прежде, чем что-то устанавливать давайте немного пробежимся по истории языка и узнаем, какие версии у него были. 

\begin{itemize}
    \item \textbf{C с классами.} Изначально был язык программирования C, но в нём не было поддержки нужных для Бьёрна Страуструпа технологий, поэтому он придумал как усовершенствовать язык C. Так в 1983-ем году и началась история C++.
    \item \textbf{C++98.} Далее язык долго развивался и в 1998-ом году вышел его первый стандарт. Во всех стандартах C++ описывается какой функционал есть в языке и как он работает. Никакая программа, способная обрабатывать код на C++ в стандарт не входит.
    \item \textbf{C++11.} Далее язык снова долго развивался и вышел стандарт с обновлениями: изменили \term{стандартную библиотеку} (в ней находятся ввод-вывод информации, контейнеры, некоторые алгоритмы и т.д.) и \term{ядро языка} (добавили отдельные циклы по контейнерам). В целом на этом стандарте уже можно писать олимпиадные программы.
    \item \textbf{C++14.} Через 3 года вышел стандарт, устранивший ошибки предыдущего. Также стандарт немного расширил функционал.
    \item \textbf{C++17.} Следующий стандарт снова добавил функционал к языку и некоторые синтаксические средства (например \term{структурное связывание}), пример использования такого синтаксиса в книге есть.
    \item \textbf{C++20.} Этот стандарт добавил удобный \term{оператор трёхстороннего сравнения}, про него также написано в этой книге.
    \item \textbf{C++23.} И самый новый стандарт на текущий момент был выпущен в 2023-ом году, в этом стандарте мне не известен синтаксис, который можно было бы применять в олимпиадах.
    \item \textbf{C++26.} Этот стандарт пока находится на стадии обсуждения и в нём могут произойти какие-то интересные нововведения.
\end{itemize}

Теперь можем перейти к установке самого Code::Blocks вместе с версией языка 2017-го года (можно установить и более новую версию, но это будет немного сложнее). Для этого переходим на \href{https://www.codeblocks.org/downloads/binaries/}{сайт Code::Blocks} и скачиваем установочный файл под названием \term{codeblocks-20.03mingw-setup.exe} (для Windows). В этом файле содержится и Code::Blocks, и компилятор для C++17, если у Вас уже что-то установлено, или другая операционная система, то выбирайте другой установочный файл.

Запускаем установочный файл, нажимаем \term{Next}, \term{I Agree} (конечно же, читаем, что там написано). Далее выбираем нужные компоненты, вполне подойдёт \term{Full} (занимает немного места и точно установится всё необходимое), нажимаем \term{Next}. Выбираем папку для установки (расположение по умолчанию вполне подходит), жмём \term{Install}. Ждём, пока всё установится, отказываемся от запуска Code::Blocks, нажимаем \term{Next} и \term{Finish}. 

Запускаем IDE (ярлык автоматически добавляется на рабочий стол), в окне \term{File associations} выбираем \term{Yes, associate Code::Blocks with C/C++ file types}, чтобы по умолчанию файлы открывались в IDE, жмём \term{OK}.

Теперь создадим новый проект, в котором мы и будем писать наши программы на С++. Для этого жмём \term{Create a new project}, \term{Console application}, \term{Next}, выбираем C++ и жмём \term{Next}. Теперь выбираем место, где будет храниться папка с нашим проектом (\term{Folder to create project in} и указываем название проекта (\term{Project title}), жмём \term{Next} и \term{Finish}.

Теперь у нас есть проект, в котором мы можем работать. На верхней панели самые важные кнопки: жёлтая шестерёнка — сборка проекта и зелёный треугольник — его запуск. Остаётся лишь сказать, что во вкладке \term{Settings} > \term{Compiler} > \term{Compiler settings} > \term{Compiler Flags} можно выбрать версию C++ (например, для C++17 ставим галочку около \term{... C++17 ISO C++ ...}).

Когда захотите закрыть проект, то откроется окно с сохранением изменений, в нём жмём \term{Yes}. Чтобы потом заново открыть проект, достаточно будет открыть файл с расширением \term{*.cbp}, который хранится в папке с проектом.

Возможно, стоит добавить файлы Code::Blocks в Path, чтобы можно было компилировать программы вручную. Для этого ищем \term{Изменение системных переменных среды}, \term{Переменные среды}. Далее выбираем (для одного пользователя, или же для всего компьютера) \term{Path}, \term{Изменить}. Далее жмём \term{Создать}, вставляем путь \term{<Путь до Code::Blocks> \textbackslash MinGW \textbackslash bin} и закрываем окно кнопкой \term{OK}. Далее ещё раз жмём \term{Создать}, вставляем путь \term{<Путь до Code::Blocks>} и закрываем все окна кнопкой \term{OK}.

Так мы получим Code::Blocks, установленный на наш компьютер, и значит, скоро сможем перейти непосредственно к программированию.
