\head{Дебаг}
Случается, что код написали, а он почему-то не работает. В таких случаях приходится его в каком-то виде дебажить. Упомяну про два очень известных способа искать ошибки в своём коде, и один не дружелюбный, зато максимально гибкий.

Первый известный способ -- добавить так называемые \term{debug print} -- вывод переменных, информации о запуске программы и т.п. в консоль. У такого способа понятны минусы -- во-первых нужно заранее угадать в каких местах нужно делать выводы, чтобы понять что сломалось; во-вторых, чтобы сделать выводы в нужных местах, придётся пересобрать программу; и в третьих, нет функционала прервать программу в какой-то точке выполнения, можно только дождаться её завершения. В целом для прерываний в произвольном месте можно добавить \term{debug input}. Но если вы всё же решили идти по этому пути, то в самом начале коде можно добавить строку:

\cpp{infra/debug}{1}

И потом каждый вывод окружать макросами:

\cpp{infra/debug}{3}

Или же можно сделать чуть более хитрую конструкцию из макросов:

\cpp{infra/debug}{5}

И каждый раз делать вывод как:

\cpp{infra/debug}{1}

Цель всех этих макросов в том, чтобы можно было отключить все ненужные выводы закомментировав всего одну строку (которая представлена в самом первом кусочке кода).

Теперь немного скажу про второй известный способ -- всякие дебагеры, встроенные в IDE. Сам я ими никогда не пользовался, но потенциально, если научиться, то ошибки могут искаться быстрее чем прошлым методом. В любом случае, это точно более масштабируемое и переносимое решение.

И теперь немного о грустном. В реальной жизни не всегда существует IDE и графический интерфейс. Поэтому приходится пользоваться консольным дебагером \term{gdb}. По нему, конечно же, всего не расскажешь в короткой заметке, но вот маленький список его команд:

\starttable
\begin{tabular}{|c|c|c|}
\hline
Команда & Альтернатива & Эффект \\
\hline
\lsh{g++ main.cpp -o ./solve -g} & & Компиляция, дебагеру нужен флаг \lsh{-g}. \\
\hline
\lsh{gdb ./solve} & & Запустить дебагер \\
\hline
\lsh{b main} & break & Создать точку останова на функции \lsh{main} \\
\lsh{r} & run & Запустить до первой точки останова \\
\lsh{p var} & print & Вывести переменную \lsh{var} (можно контейнеры) \\
\lsh{c} & continue & Продолжить до следующей точки останова \\
\lsh{s} & step & Следующая инструкция, входит в функции \\
\lsh{n} & next & Следующая инструкция, не входит в функции \\
\lsh{q} & quit & Выключить дебагер \\
\hline
\end{tabular}
\endtable
