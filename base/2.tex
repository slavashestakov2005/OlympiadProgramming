\head{Типы данных}
В этом блоке мы изучим базовое понятие для языка C++ — \term{типы данных}. Если Вы раньше изучали языки программирования, в которых указывать типы данных не обязательно (например, тот же Python), то Вам придётся привыкать, потому что типизация в C++ статическая, а значит, у всех переменных придётся указывать их тип.

Тип данных — это то значение, которое может хранить переменная. Например, какая-то переменная может хранить только число, а другая — только строку. Преобразование переменной из одного типа данных в другой называется \term{привидением}. Также, кроме хранимого значения, в C++ типы данных различаются и размерами, занимаемыми в памяти, поэтому их важно изучить, чтобы использовать при необходимости на соревнованиях. Список основных типов данных вы можете посмотреть ниже.

\starttable
\begin{tabular}{|c|c|c|}
\hline
Название & Размер в байтах & Хранимое значение \\
\hline
\lcpp{bool} & 1 & \lcpp{true} (истина) или \lcpp{false} (ложь) \\
\lcpp{char} & 1 & Хранит один символ, приводится к числу. \\
\lcpp{short int} & 2 & Целое число из диапазона $[-2^{15}; 2^{15}-1]$ \\
\lcpp{long int} или \lcpp{int} & 4 & Целое число из диапазона $[-2^{31}; 2^{31}-1]$ \\
\lcpp{long long int} & 8 & Целое число из диапазона $[-2^{63}; 2^{63}-1]$ \\
\lcpp{float} & 4 & Рациональное число из диапазона $[-2^{31}; 2^{31}-1]$ \\
\lcpp{double} & 8 & Рациональное число из диапазона $[-2^{63}; 2^{63}-1]$ \\
\hline
\end{tabular}
\endtable

Важный нюанс в C++ это то, что в процессе своего исторического развития C++ много пережил и работал под разными архитектурами процессоров, поэтому получилось столько разных типов данных, и \lcpp{long int} случайно совпал с \lcpp{int}. Вообще-то, согласно стандарту C++, размер для типа \lcpp{int} не зафиксирован, но во всех олимпиадах размер \lcpp{int} именно 4 байта.

Также во всех типах, где есть слово \lcpp{int}, его писать не обязательно (кроме обычного \lcpp{int}, разумеется). Ещё у всех типов, кроме \lcpp{bool} есть их беззнаковый аналог, для его использования добавляется \lcpp{unsigned} перед названием типа (\lcpp{unsigned short}, например). И диапазон значений у таких типов сдвигается в неотрицательные значения (например, \lcpp{unsigned short} хранит значения из диапазона $[0; 2^{16}-1]$).

Теперь обсудим использование числовых типов данных в олимпиадной информатике. Здесь стараются использовать только целые числа (потому что при работе с рациональными возникают погрешности), поэтому самый популярный тип — \lcpp{int}. Если его диапазона не хватает, то используют \lcpp{long long}. Если рациональных чисел избежать не удаётся, то используют \lcpp{double}, потому что он точнее, чем \lcpp{float}. Остальные типы используются реже и в задачах со специальными ограничениями.

Чтобы понять, какой тип данных понадобится — нужно внимательно прочитать ограничения входных данных к задаче и посмотреть, на сколько большие значения будут возникать в процессе работы Вашего алгоритма. Для этого можно использовать приближения: $2^{31} \approx 2 \cdot 10^9$ и $2^{63} \approx 9 \cdot 10^{18}$. Поэтому понятно, что, например, для \sortme{40}{этой задачи} обычного \lcpp{long long} не хватит, так как ограничение на входное число до $10^{19}$, но $10^{19} > 2^{63} - 1$, а значит, нужно или использовать его \lcpp{unsigned} версию, или, всё же, воспользоваться рациональными числами.

Теперь, зная типы данных, перейдём к \term{литералам}. Здесь всё просто, потому что литерал — это константа, включённая в код программы, и для этого не нужно делать ничего особенного. Вот так в C++ выглядят литералы:

\cpp{2}{7}

Что же касается использования новых знаний, то здесь всё просто. Вы всегда перед первым использованием переменной должны указать её тип, а указывается он перед названием переменной. Это называется \term{объявлением переменной} и выглядит например так:

\cpp{2}{1}

Если Вам нужно этой переменной сразу присвоить какое-то значение, то это называется \term{определением}, и нужно написать знак равенства и какой-нибудь литерал, другую переменную или какое-нибудь выражение. Делается это вот так:

\cpp{2}{3}

При этом, если нужно сразу несколько переменных, то их можно перечислить через запятую:

\cpp{2}{1}
