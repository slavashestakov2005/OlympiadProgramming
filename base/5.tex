\head{Строки и ввод-вывод}
Теперь, когда мы узнали синтаксис С++, можно переходить и к изучению чуть менее важных вещей. В этом блоке мы познакомимся со строками и операциями ввода-вывода.

Строки в C++ пишутся в двойных кавычках (\lcpp{""}) и являются контейнером для символов. Но, как уже отмечалось в предыдущих блоках, символы кодируются согласно таблице ASCII, а в этой таблице есть управляющие символы. Например, таким является символ переноса строки — его не видно, но при этом он говорит, что продолжать нужно с новой строки. В C++ эти символы тоже поддерживаются и начинаются они с \lcpp{\}. Все управляющие символы можно увидеть в табличке ниже.

\starttable
\begin{tabular}{|c|c|}
\hline
Символ & Описание \\
\hline
\lcpp{'\r'} & возврат каретки в начало строки \\
\lcpp{'\n'} & новая строка \\
\lcpp{'\t'} & горизонтальная табуляция \\
\lcpp{'\v'} & вертикальная табуляция \\
\lcpp{'\"'} & двойные кавычки \\
\lcpp{'\''} & апостроф \\
\lcpp{'\\'} & обратный слеш \\
\lcpp{'\0'} & нулевой символ \\
\lcpp{'\?'} & знак вопроса \\
\lcpp{'\a'} & сигнал бипера (спикера) компьютера \\
\hline
\end{tabular}
\endtable

Из всех специальных символы самый полезный (популярный в олимпиадном программировании) — это перенос строки, поэтому приведём пример его использования:

\cpp{5}{1}

Данная строка кода будем выводить надпись в две строки: на первой - \lcpp{"Hello"}. а на второй — \lcpp{"world!"}. Другие управляющие символы используются аналогично.

Скажем совсем немного про работу с символами. Иногда бывает нужно перевести символы из верхнего регистра в нижний и наоборот. Это можно сделать формулой, или же использовать две встроенные функции:

\cpp{5}{2}

Теперь перейдём к самим строкам. Как упоминалось раньше, они в C++ заключаются в двойные кавычки (и именно так у нас в предыдущей программе). Также мы знаем, что у всего в C++ есть свой тип данных. Так какой же тип нам использовать для строк? Оказывается, что такой тип данных встроен (как и все важные и достаточно универсальные типы), и называется он \lcpp{std::string}. Также, как было сказано в предыдущих блоках, если мы не хотим постоянно писать префикс \lcpp{std::}, то достаточно один раз в программе написать \lcpp{using namespace std;}. И последний нюанс со строками, для их использования нужно подключить специальный файл:

\cpp{5}{1}

А теперь, когда мы знаем всю нюансы, то уже можем что-нибудь со строками сделать, например написать программу-приветствие:

\cpp{5}{9}

Но, конечно же, это не всё, что можно делать со строками. Во-первых, строки можно друг с другом \term{конкатенировать}, то есть дописывать одну строку в конец другой. Делается это с помощью оператора \lcpp{+}:

\cpp{5}{3}

Также у строк есть длина и проверка на пустоту:

\cpp{5}{2}

Получение символа по индексу и получение подстрок:

\cpp{5}{2}

Ещё можно менять саму строку (добавлять символы в конец или в любое другое место; удалить какие-нибудь символы):

\cpp{5}{4}

Можно получать первый и последний элементы строки:

\cpp{5}{2}

Можно получать \term{итераторы} на начало и конец строки (пока может выглядеть бесполезным, но оно нам понадобится):

\cpp{5}{2}

Собственно, это все основные методы строк, но если Вам понадобятся ещё какие-нибудь другие методы, то Вы всегда можете посмотреть их в официальной документации.

Теперь можно и переходить к вводу-выводу. Для начала снова придумаем какую-нибудь задачу. Пусть нам нужно считать время (в формате <часы>:<минуты>), извлечь из них отдельно часы и минуты и потом вывести так же, как и было задано изначально.

Первый вариант, который может приходить на ум — это сначала считать строку, потом найти в ней двоеточие, и дальше из левой части получить часы, а из правой — минуты. Реализацию данного способа мы оставим на исполнение читателям.

Другой способ может появиться, если немного поэкспериментировать с оператором \lcpp{cin} и разными типами данных. Оказывается, что вот такой код будет работать корректно:

\cpp{5}{9}

Понятно, как C++ справляется с этим: он знает, что двоеточие точно не может быть в целом числе, поэтому сам делает то, что мы придумали в первом способе.

Но и это ещё не всё, потому что C++ сохранил \lcpp{scanf} для ввода и \lcpp{printf}, для вывода, которыми пользовались ещё в C. И такая же программа с этим функциями будет выглядеть вот так:

\cpp{5}{8}

Здесь возникли строки \lcpp{"%d"}. Они обозначают, что нужно считать или напечатать число с типом \lcpp{int}. Если нужен другой тип, то его можно узнать по таблице ниже.

\starttable
\begin{tabular}{|c|c|}
\hline
Символ & Тип данных \\
\hline
\lcpp{"%d"} & \lcpp{int} \\
\lcpp{"%u"} & \lcpp{unsigned int} \\
\lcpp{"%x"} & \lcpp{int} в шестнадцатеричной системе счисления \\
\lcpp{"%o"} & \lcpp{int} в восьмеричной системе счисления \\
\lcpp{"%f"} & \lcpp{float} \\
\lcpp{"%lf"} & \lcpp{double} \\
\lcpp{"%c"} & \lcpp{char} \\
\lcpp{"%s"} & \lcpp{string} \\
\hline
\end{tabular}
\endtable

Хорошо, с этой задачей вроде бы разобрались. А что делать, если нам вдруг понадобится считывать всё, что нам вводится, включая пробелы. Как тогда быть, если при считывании в \lcpp{string} попадают только не пробельные символы?

И для этого в C++ есть решение — \lcpp{getline}:

\cpp{5}{2}

Этот код запишет в переменную \lcpp{s} всё, что введёт пользователь до перевода на новую строку.

И, наконец, последнее не обычное использование ввода. Предположим, что наша задача — это считать сколько-то чисел (но мы не знаем сколько) и вывести их сумму. И тут у нас возникнет проблема: считывать числа и суммировать их мы, конечно, можем, но вот проверять, что больше числе нет — пока не умеем. Но и для этого C++ придумал выход:

\cpp{5}{10}

Причина, по которой это код работает такова: \lcpp{cin} считывает значение, сохраняет его в переменную и возвращает \lcpp{true}, если считывание удалось, иначе — \lcpp{false}.

