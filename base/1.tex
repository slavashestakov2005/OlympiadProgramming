\hypertarget{1.1}{}
\head{Первая программа}
Сегодня мы с Вами погрузимся в изучение нового для Вас языка программирования — C++, но изучать его мы будем в контексте олимпиад. Также по этой книге можно не только учить C++, но и повторять пройденный материал.

Так почему же выбор пал именно не изучение C++? Ответ очевиден — этот язык общепризнан быстрым и поэтому очень часто используется для олимпиадного программирования.

Теперь, когда мы поняли, почему выбор пал именно на этот язык, можно и начинать его изучать. Для этого открываем доступную среду разработки и вставляем в неё следующий код, представленный ниже.

\cpp{1}{9}

Читатель, знакомый с другими языками программирования, может сразу догадаться, что данная программа выведет традиционную надпись \lcpp{Hello world!}, и это действительно так!

Давайте разберёмся с кодом построчно.

В строке №1 написан однострочный комментарий (начинается с \lcpp{//} и заканчивается с концом строки кода). Он нужен только для программиста, его содержимое не выполняется.

В строке №2 мы подключаем (\lcpp{#include}) файл, в котором описаны функции по работе с вводом-выводом информации (\lcpp{iostream}). Стоит отметить, что в C++ подключение встроенных функций не требует лишних временных и вычислительных затрат, и разные функции находятся в разных файлах (например, чуть позже вы узнаете про \lcpp{cmath} для работы с математикой и \lcpp{vector} для массивов с изменяемой длиной). В связи с этим, в олимпиадном программировании принято подключать сразу всё, потому что для этого не нужно запоминать, какие функции где хранятся, и для этого достаточно написать только одну строку:

\cpp{1}{1}

Теперь разберёмся со строкой №3. В C++ для структурирования кода придуманы так называемые \term{пространства имён}, которые обозначаются ключевым словом \lcpp{namespace}. Все стандартные функции языка находятся в пространстве имён \lcpp{std}, и для их вызова нужно писать \lcpp{std::} перед названием функции (например, \lcpp{std::cout}). Но в олимпиадном программировании снова упрощают себе жизнь и пишут строку №3, чтобы каждый раз не добавлять \lcpp{std::}.

Любые программы на C++ всегда компилируются, то есть переводятся на машинный язык, и поэтому всегда в программах на C++ есть \term{точка входа} —  в простейшем случае это функция \lcpp{main}. В строке №4 как раз объявляется эта функция, с возвращаемым значением \lcpp{int} (целое число), причём такое возвращаемое значение обязательно (единственное, что можно делать — не указывать ничего вместо \lcpp{int}, тогда компилятор сам поймёт, что там должно было быть написано). Функции в олимпиадном программировании используются, поэтому мы научимся с ними работать, но это будет чуть позже. Пока важно лишь понимать, что тело функции ограничивается фигурными скобками.

Далее идёт строка с выводом надписи \lcpp{Hello world!}. Если вы смогли понять значение кода выше, то с этой строкой кода точно справитесь — все строковые значения в C++ всегда содержатся в двойных кавычках, а вывод осуществляется с помощью \lcpp{cout} и операции сдвига влево (\lcpp{<<}, в простонародье — «ёлочка влево»). Слово \lcpp{endl} обозначает, что после этой надписи нужно вывести перевод строки.

Строка №6 возвращает значение из функции. Казалось бы, кому его возвращать? Но нет - оно нужно, и используется компьютером, запустившим программу. При этом, договорились, что если программа вернула 0 или ничего не вернула, то считается, что она выполнилась успешно (поэтому на самом деле строка \lcpp{return 0;} — не обязательна). А если программа вернула из функции \lcpp{main} не 0, то это считается за ошибку во время выполнения. Поэтому на всех олимпиадах если программа вернула не 0, то считается, что она сломалась, и участник олимпиады получает соответствующий вердикт.

На строке №7 у нас заканчивается функция \lcpp{main}, поэтому там просто ставится закрывающая фигурная скобка.

В строках 8-9 показано, как создавать многострочные комментарии, они начинаются с \lcpp{/*} и заканчиваются \lcpp{*/}.

Таким образом, общий вид олимпиадной программы на C++ таков:

\cpp{1}{6}

Может показаться, что такой код слишком длинный, но на самом деле цель (написать быструю программу на олимпиаде) оправдывает средства!
