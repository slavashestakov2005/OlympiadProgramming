\head{Динамическое программирование}
Как мы выяснили раньше, перебор работает слишком долго, а жадный алгоритм может не работать. В таких случаях может помочь \term{динамическое программирование}. Основная идея в том, что, задача для заданного $n_0$ может легко решаться, если мы уже умеем решать задачи для меньших $n$. Тогда мы можем решить серию таких задач, при этом запомнив ответы, и потом получить искомый ответ.

При этом в зависимости от того, в каком порядке мы решаем задачи, разделяют \term{восходящую} и \term{нисходящую} динамики. В восходящей задачи решаются от меньший к большей, а в нисходящий — наоборот, от большей задачи идут вниз, к меньшим.

Раньше мы уже немного использовали динамическое программирование (там мы считали факториалы и числа Фибоначчи), но не называли его так. Теперь же мы посмотрим на его применение и в более сложных случаях.


\subhead{Числа Фибоначчи}
Для начала возьмём простой пример — числа Фибоначчи. Их определение таково: $F_0 = 0, F_1 = 1, F_n = F_{n - 2} + F_{n - 1}$, при $n \ge 2$. И требуется посчитать $F_n$.

Здесь у нас нам дано \term{рекуррентное соотношение} — следующий элемент последовательности выражен через предыдущий, а это как раз то, чего мы и хотим от динамического программирования. А это значит, что мы сейчас придумаем два динамических решения (восходящее и нисходящее).

Начнём с восходящей динамики. Для начала давайте создадим структуру, в которой будем хранить ответы (в этой задаче достаточно обычного массива или \lcpp{vector}'а, но в более сложных задачах может потребоваться и что-то более сложное: многомерные массивы, или \lcpp{map}, например). И теперь занесём в неё те ответы, которые мы точно знаем (это будет база для нашей динамики, от неё будут считаться другие значения), в нашем случае это $F_0 = 0$ и $F_1 = 1$. А теперь будем перебирать в восходящем порядке все значения $2 \leq i \leq n$, для каждого из них считать ответ по известное формуле $F_i = F_{i - 2} + F_{i - 1}$. При этом предыдущие ответы ($F_{i - 2}$ и $F_{i - 1}$) мы будем брать из контейнера, где мы храним ответы, а новый ответ — тоже будем записывать в этот контейнер. Так у нас получится посчитать $F_i$ для всех $0 \leq i \leq n$ и мы решим исходную задачу.

Теперь разберёмся с нисходящим решением. Начинаться оно будет точно также - создадим контейнер для ответов и сохраним в него $F_0$ и $F_!$. А для вычисления $F_n$ мы будем использовать рекурсивную функцию от одного аргумента $i$. Она будет работать так: если мы раньше считали $F_i$, то мы можем вернуть этот ответ, взяв его из контейнера. А иначе мы посчитаем наше $F_i$, вызвав эту же функцию рекурсивно (два раза: от $i - 2$ и от $i - 1$) и сложив два результата. Как только мы узнали $F_i$, то можем его записать в контейнер с ответами и вернуть вычисленное значение. Теперь, чтобы получить $F_n$ будет достаточно вызвать нашу функцию с аргументом $n$ и так мы получим нужное значение. При этом, так же, как и в прошлом решении, у нас за одно посчитаются и все остальные $F_i$ для $0 \leq i \leq n$, и потом можно их как-то дополнительно использовать.

Так мы довольно подробно разобрали решения для этой конкретной задачи, его сложность, кстати, составляет \O{n}. Но в целом, все остальные задачи с динамическим программированием будут решаться точно также: сначала запоминаются базовые ответы, а потом вычисляются все остальные. Поэтому другие задачи мы не будем разбирать столь подробно, а будем лишь находить базовые ответы и способ вычислить следующий ответ через предыдущий\footnote{В математике названия немного другие: динамическое программирование называется \term{индукцией}, базовые ответы — \term{базой индукции}, получение следующего ответа через предыдущие — \term{шаг индукции}.
}. 

Для полноты требуется сказать, что Числа Фибоначчи можно вычислять за \O{1}, если использовать \term{Формулу Бине}:
$$ F_n = \frac{ \left(\frac{1 + \sqrt{5}}{2}\right)^n - \left(\frac{1 - \sqrt{5}}{2}\right)^n }{\sqrt{5}} $$

Ну что ж, докажем эту формулу.

\begin{enumerate}
    \item Рассмотрим последовательности, которые одновременно отвечают двум условиям: $f_n = q \cdot f_{n - 1}$ (геометрическая прогрессия) и $f_n = f_{n - 2} + f_{n - 1}$ (фибоначчиевая последовательность).
    \item Такая совокупность условий эквивалентна $f_0 \cdot q^n = f_0 \cdot q^{n - 2} + f_0 \cdot q^{n - 1}$, её можно преобразовать в $q^2 = q^1 + 1$. А решениями такого уравнения являются $q_{1, 2} = \frac{1 \pm \sqrt{5}}{2}$. Переобозначим их за $\lambda_1 = \frac{1 - \sqrt{5}}{2}$ и $\lambda_2 = \frac{1 + \sqrt{5}}{2}$.
    \item Теперь заметим, что если есть две фибоначчиевые последовательности ($a_n = a_{n - 2} + a_{n - 1}$ и $b_n = b_{n - 2} + b_{n - 1}$), то мы можем их сложить ($c_n = a_n + b_n$) и получить новую фибоначчиевую последовательность ($a_n + b_n = c_n = c_{n - 2} + c_{n - 1} = a_{n - 2} + b_{n - 2} + a_{n - 1} + b_{n - 1}$). Также фибоначчиевая последовательность получается, если все элементы умножить на константу ($d \cdot a_n = d \cdot a_{n - 2} + d \cdot a_{n - 1}$).
    \item А теперь этими двумя операциями получим числа Фибоначчи:
        $\begin{cases}
        F_0=0=c_1 + c_2 \\
        F_1=1=c_1 \cdot \lambda_1 + c_2 \cdot \lambda_2 \\
        F_n= c_1 \cdot \lambda_1^n + c_2 \cdot \lambda_2^n
        \end{cases}$.
    Если решить эту систему, то получится $c_1 = -\frac{1}{\sqrt{5}}$ и $c_2 = \frac{1}{\sqrt{5}}$.
    \item А теперь эти $\lambda_1, \lambda_2, c_1, c_2$ можно подставить в формулу $F_n= c_1 \cdot \lambda_1^n + c_2 \cdot \lambda_2^n$ и получить формулу Бине:
    $$ F_n = \frac{ \left(\frac{1 + \sqrt{5}}{2}\right)^n - \left(\frac{1 - \sqrt{5}}{2}\right)^n }{\sqrt{5}} $$
\end{enumerate}


\subhead{Задача о замощении полосы}
Пусть у нас есть полоса размера $2 \times n$ (2 - высота, $n$ - ширина) и её нужно замостить прямоугольниками $1 \times 2$ (прямоугольники можно поворачивать, но нельзя накладывать друг на друга; вся полоса должна быть покрыта, прямоугольники не могут вылазить за пределы полосы). Требуется посчитать количество таких замощений.

Может возникнуть желание перебрать все варианты замощений, но давайте не будем этого делать, потому что такое способ будет работать только для не больших значений $n$. Так что будем придумывать динамическое решение, обозначив $d_i$ за ответ для полосы $2 \times i$. Рассмотрим полосу $2 \times i$ и подумаем, что может быть на её конце (мы ведь хотим перейти к меньшей полосе, а для этого достаточно убрать несколько прямоугольников с края). Если у нас на конце находится вертикальный прямоугольник, то оставшуюся часть можно было составить $d_{i - 1}$ способом. Если же там горизонтальный прямоугольник, то их должно быть два (один над другим, так как иначе не получится закрыть обе крайние клетки), а значит заполнить оставшуюся часть полосы можно $d_{i - 2}$ способами.

Теперь остаётся только собрать всё вместе. Мы рассмотрели два варианта (вертикальный и горизонтальный), это разный и независимые варианты, поэтому их нужно сложить: $d_i = d_{i - 1} + d_{i - 2}$. Остаётся определить базу для вычисления других значений. Видимо, достаточно будет два значения (так как в формуле мы используем два предыдущий) поэтому возьмём $d_0 = 0$ и $d_1 = 1$. Теперь, во-первых, это можно написать, а во-вторых, можно заметить, что это формула для чисел Фибоначчи.

Немного модифицируем задачу, теперь у нас полоса $3 \times n$ и прямоугольники $1 \times 3$, а все остальные условия те же.

Снова рассмотрим конец полосы $3 \times i$: если там вертикальный прямоугольник, то у нас есть $d_{i - 1}$ способов его дополнить до полосы; если же там горизонтальный прямоугольник, то их должно быть три и они замостят квадратик $3 \times 3$, поэтому дополнить их можно будет $d_{i - 3}$ способами. Эти варианты независимы, как и в предыдущей задаче, поэтому формула будет: $d_i = d_{i - 1} + d_{i - 3}$. Остаётся лишь определиться с начальными значениями: $d_0 = 0$, $d_1 = 1$, $d_2 = 2$ (нам нужно три значения, так как в формуле есть $i - 3$). И теперь это можно легко написать.

Заметим, что сложность алгоритма для обеих задач будет \O{n}, хотя если взять полосу и прямоугольник произвольных размеров, то вычислять станет заметно сложнее, так как вариантов взаимного расположения прямоугольников станет больше.

Но для части размеров существует формула. которая позволяет посчитать количество замощений за \O{n \cdot m}. Доказана она была в 1961 году, и работает для досок любого размера $n \times m$, которые замощаются прямоугольниками $1 \times 2$:

$$\prod\limits_{a=1}^{\lceil {\frac{n}{2}} \rceil} \prod\limits_{b=1}^{\lceil {\frac{m}{2}} \rceil} 4 \cdot \left(\cos^2\frac{\pi a}{n + 1} + \cos^2\frac{\pi b}{m + 1} \right)$$

Эту формулу мы, пожалуй, оставим без доказательства :)


\subhead{Задача про хромую ладью}
Теперь рассмотрим другую задачу: есть поле $n \times m$ ($n$ строк и $m$ столбцов), в каждой ячейке стоит число $a_{i, j}$. Хромая ладья стоит в левой верхней клетки и может ходить только вправо или вниз на 1 клетку; она идёт в правую нижнюю клетку и по дороге складывает все числа в клетках, которые посетила. Нужно посчитать, какую максимальную сумму может набрать ладья.

Теперь попробуем для каждой клетки посчитать $d_{i, j}$ ($1 \leq i \leq n, 1 \leq j \leq m$)  — максимальную сумму, которую можно набрать, если идти от этой клетки в правую нижнюю. Во-первых понятно, что в конечной клетки идти некуда, поэтому $d_{n, m} = a_{n, m}$. Во-вторых из правого столбца и из нижней строки есть только один вариант хода, поэтому $d_{i, m} = a_{i, m} + d_{i + 1, m}$ и $d_{n, j} = a_{n, j} + d_{n, j + 1}$. Остаётся понять, как посчитать ответы для оставшихся клеток. Понятно, что для этого достаточно из текущей клетки ($i, j$) пойти туда, откуда можно набрать большую сумму, поэтому $d_{i, j} = a_{i, j} + \max(d_{i, j + 1}, d_{i + 1, j})$.

Теперь у нас есть формулы, которые позволят вычислить все значения $d_{i, j}$ и базовые ответы для нашей динамики — $d_{n, m}$, $d_{i, m}$ и $d_{n, j}$. А значит остаётся только аккуратно написать программу. Если использовать нисходящую динамику с рекурсией, то рекурсия сама вычислит значения в нужном порядке, поэтому здесь задумываться особо не придётся. Но если же использовать восходящую динамику, то нужно будет аккуратно работать с индексами, чтобы следующее значение вычислялось только тогда, когда вычислены предыдущие для него; но в этой задаче понятно, в каком порядке можно перебирать: будем смотреть на все строки снизу вверх, а на элементы в строках справа налево — так мы как раз получим, что для очередной клетки уже известны ответы для её соседей справа и снизу.

Сложность такого алгоритма будет \O{n \cdot m}, ведь нам ровно столько ответов $d_{i, j}$ нужно будет посчитать.


\subhead{Задача о размене}
В жадных алгоритмах нам встретилась задача о размене, но теперь мы сможем её решить! Напомним её условие: имеется $n$ номиналов монет $a_1, a_2, \ldots, a_n$, количество монет каждого наминала не ограничено. Требуется набрать сумму $x$ наименьшим числом монет.

Давайте действовать так: пусть нам нужно набрать сумму $s$, тогда переберём все номиналы и попробуем набрать сумму $s - a_i$ наименьшим количеством монет. Потом выберем минимум из этих вариантов и не забудем прибавить $1$, ведь мы использовали монету $a_i$. Понятно, что такой алгоритм действительно правильно работает, ведь сумму чем-то нужно набирать, а мы перебрали все варианты для начала набора этой суммы.

Теперь, когда у нас есть алгоритм, давайте попробуем получить формулы. Пусть $d_i$ — наименьшее количество монет, которыми можно набрать сумму $i$. Тогда основная формула, которую мы описали выше будет вот такой: $d_i = 1 + \min\limits_{1 \leq j \leq n} d_{i - a_j}$. Остаётся лишь определиться с базовых ответов для нашей динамики. Поскольку здесь у нас из индекса вычитается какое-то число из входных данных, то не понятно, сколько должно быть базовых ответов. Но давайте посмотрим на тот момент, когда мы должны закончить набирать такую сумму, потому что точно знаем, какой будет оптимальный ответ. Во-первых, при вычитании могла получиться отрицательная сумма, но понятно, что для этой суммы размена нет, поэтому $d_i = \infty$ для $i < 0$. Во-вторых, мог получиться 0, это вполне нормальное число, которое набирается без монет, поэтому $d_0 = 0$. И в-третьих, могла получиться положительная сумма, но для неё ничего не понятно, поэтому будем продолжать вычитания.

На этой задаче можно заметить, что для ответа могут понадобиться не все числа. В самом деле: пусть $n = 3,\ a = [5, 15, 25]$. В этом примере у нас номиналы всех монет делятся на $5$, поэтому, чтобы найти ответ для $x$ нам потребуются только ответы для $x - 5, x - 10, x - 15, \ldots, x \mod 5$. Поэтому для данной задачи восходящая динамика пишется почти также, как и в прошлых задачах: изначально сделаем $d_i = \infty$, $i > 0$; потом из всех сумм $s$ из диапазона $[0; x)$ сделаем обновление для чисел, которые можно получить ($d_{s + a_i} = \min( d_{s + a_i}, 1 + d_{s} )$). Но при этом придётся получать ответы для всех сумм. Если же мы будем использовать нисходящую рекурсию, то мы будем переходить только в те варианты, которые полезны для получения нашей суммы, а таких вариантов потенциально может быть меньше (во всяком случае не больше).

Как и в предыдущей задаче (да и во многих других задачах на динамику), сложность такого алгоритма будет определяться количеством ответов, которые нужно получить, то есть будет \O{n \cdot x}.


\subhead{Дискретная задача о рюкзаке}
Задача о дискретном рюкзаке очень похожа на непрерывный рюкзак, с той лишь разницей, что вещи не делимы. Итоговая формулировка такова: вор пробрался на склад, где имеется $n$ вещей стоимостью $c_i$ и массой $m_i$. Нужно выбрать такие вещи, чтобы их суммарная масса не превосходила заданного $w$ (вместимость рюкзака), и при этом набрать максимальную стоимость. Вещи не делимы.

Здесь, к сожалению, жадный алгоритм перестаёт работать, ведь можно придумать такой пример: $n = 2,\ c = [10, 12],\ m = [2, 3],\ w = 4$. Жадное решение (которые выбирает наибольшее значение $\frac{c_i}{m_i}$) выберет первый предмет, тогда как второй — лучше. Поэтому давайте придумаем динамическое решение.

Будем вычислять $d_{i, j}$ — наибольшая стоимость, которую можно набрать, если использовать только первые $i$ предметов, при этом их суммарная масса составляет $j$. Теперь подумаем над переходом от одного значения к другому. Очередной предмет $i$ можно или взять, или не взять, поэтому $d_{i, j} = \max( d_{i - 1, j}, d_{i - 1, j - m_i} + c_i)$. При этом набирать отрицательную массу нельзя, поэтому $d_{i, j} = -\infty$, если $j < 0$. Также нулевую массу можно взять всегда, поэтому $d_{i, j} = 0$, если $j = 0$.

Раз мы знаем формулу, то остаётся лишь посчитать $d_{i, j}$. Делать это можно любым изученным способом, массы — тоже можно перебирать в любом порядке, главное — перебирать предметы в порядке $1, 2, \ldots, n$.

Как это и не удивительно, но сложность представленного выше решения — \O{n \cdot m}.


\subhead{Наибольшая общая подпоследовательность}
И, напоследок, разберём ещё одну задачу, в которой у нас есть два набора $a_1, a_2, \ldots, a_n$ и $b_1, b_2, \ldots, b_m$. От нас требуется найти общую подпоследовательность наибольшей длины (подпоследовательность — часть элементов набора, при этом нельзя менять порядок элементов).

Раз у нас два набора, то будем делать динамику по двум переменным — индексу в первом наборе и во втором наборе соответственно, то есть у нас будет $d_{i, j}$ — наибольшая длина общей подпоследовательности, если рассмотреть только первые $i$ элементов из $a$ и $j$ первых элементов из $b$. Теперь определимся с базовыми ответами: если из какого-то набора ничего не рассматривать, то ответ точно 0: $d_{i, j} = 0$, если $i = 0$ или $j = 0$. Теперь подумаем, как сделать переход в $d_{i, j}$: для этого посмотрим на элементы $a_i$ и $b_j$. Если они равны, то кандидатом на ответ будет $d_{i - 1, j - 1} + 1$. Но также, вполне может оказаться, что для предыдущей пары индексов ответ был лучше, поэтому ещё кандидатами будут $d_{i - 1, j}$ и $d_{i, j - 1}$. И из этих ответов остаётся выбрать максимум — он и будет $d_{i, j}$.

Теперь мы знаем базовые ответы, алгоритм того, как делать переход, а значит сможем найти ответ на исходную задачу, перебрав индексы от меньшего к большему. И итоговая сложность составит \O{n \cdot m}.
