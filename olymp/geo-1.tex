\head{Геометрия: введение}
Сегодня мы начинаем изучение задач, связанных с вычислительной геометрией. Это не обычные геометрические задачи по математике, в которых нужно использовать разные теоремы, чтобы доказать что-то в задаче, а задачи, в которых нужно что-то посчитать. Например, найти площадь, найти точку, уравнение прямой и многое другое.

Сегодня мы рассмотрим такие базовые понятия вычислительной геометрии, как точки (\lcpp{Point}), вектора (\lcpp{Vector}) и прямые (\lcpp{Line}). Всё это мы будем рассматривать на плоскости, потому что задачи про трёхмерное пространство не встречаются. Вычислительная геометрия — это как раз один из тех разделов, в которых будет не очень удобно использовать структуры, поэтому мы научимся работать с ними ещё лучше. Также в вычислительной геометрии очень важна точность, поэтому мы будем использовать шаблоны, чтобы получить реализацию и для целых чисел, и для вещественных.


\subhead{Точки}
Точка очень простой геометрический объект, поэтому каждая точка будет обладать своими координатами (\lcpp{x} и \lcpp{y}), и точки будут поддерживать следующие операции (их достаточно мало):

\begin{itemize}
    \item Создание точки по координатам или без них. Для этого будем использовать конструктор с точкой \p{0}{0} по умолчанию.
    \item Сдвиг точки на вектор. Для этого определим операторы \lcpp{+} и \lcpp{-}, чтобы можно было писать \lcpp{p + v} и \lcpp{p - v}.
    \item Ввод и вывод точки. Определим операторы \lcpp{>>} и \lcpp{<<} с нужными аргументами и C++ сам поймёт, как обрабатывать выражения \lcpp{cin >> p} и \lcpp{cout << p}.
\end{itemize}

Все эти операции можно реализовать следующим образом:

\cpp{geo-1}{11}


\subhead{Вектора}
Хоть вектора, также, как и точки, хранят только два числа (\lcpp{x} и \lcpp{y}), но это уже более сложный объект, поэтому для них нам нужно определить много операций:

\begin{itemize}
    \item Создание вектора по координатам или без них; создание вектора по двум точкам. Для этого будем использовать конструкторы и вектор $\v{(0, 0)}$ по умолчанию.
    \item Сложение и вычитание векторов. Нам помогут бинарные операторы \lcpp{+} и \lcpp{-} чтобы писать \lcpp{v + u} и \lcpp{v - u}, а также их унарные версии, чтобы делать \lcpp{+v} и \lcpp{-v} (как бинарные операции, в которых второй операнд $\v{0}$).
    \item Умножение вектора на число реализуем с помощью операторов \lcpp{*} и \lcpp{/}.
    \item На произведениях векторов нужно остановиться подробнее:
        \begin{enumerate}
            \item \textbf{Скалярное произведение.} В математике обозначается как $\v{v} \cdot \v{u}$ и имеет важное равенство $|v| \cdot |u| \cdot \cos(\widehat{v;u}) = \v{v} \cdot \v{u} = v_x u_x + v_y u_y$. Считать мы, конечно, будем по второй формуле и обозначим такое произведение за \lcpp{v * u}. Также важно знать, что если $\v{v}, \v{u} \ne \v{0}$ и $\v{v} \cdot \v{u} = 0$. то $\v{v} \perp \v{u}$.
            \item \textbf{Псевдоскалярное произведение.} В математике обозначается как $\v{v} \wedge \v{u}$ и имеет важное равенство $|v| \cdot |u| \cdot \sin(\widehat{v;u}) = \v{v} \wedge \v{u} = v_x u_y - v_y u_x$. Считать мы, конечно, будем по второй формуле и обозначим такое произведение за \lcpp{v % u} (потому что хочется, чтобы приоритет произведений был одинаковым, а использовать $/$ как-то странно для произведения). Также важно знать, что если $\v{v}, \v{u} \ne \v{0}$ и $\v{v} \wedge \v{u} = 0$. то $\v{v} \parallel \v{u}$.
        \end{enumerate}
    \item Угол между векторами, обозначается как $\widehat{v;u}$. Понятно, что для этой операции можно придумать формулы через произведения: $\widehat{v;u} = \arccos{ \frac{\v{v} \cdot \v{u}}{|v| \cdot |u|} } = \arcsin{ \frac{\v{v} \wedge \v{u}}{|v| \cdot |u|} }$. Но тогда у нас возникнут дополнительный погрешности при делении (да и случай деления на ноль придётся разбирать отдельно), поэтому принято считать по формуле $\widehat{v;u} = \arctg{ \frac{\v{v} \wedge \v{u}}{\v{v} \cdot \v{u}} }$. А в программировании есть функция $\atantwo(y, x) = \arctg{\frac{y}{x}}$, которая дополнительно умеет обрабатывать случай $x = 0$. Для угла будем использовать оператор \lcpp{^}, чтобы писать \lcpp{v ^ u}.
    \item Длину вектора мы будем считать по теореме Пифагора: $len = \sqrt{x^2 + y^2}$. При этом в некоторых задачах нам достаточно использовать квадрат длины (например, если мы сравниваем длины векторов), поэтому сделаем отдельную функцию, которая бы не извлекала корень, потому что так вычисления точнее: $sqlen = x^2 + y^2$.
    \item Полярный угол будем считать через $\atantwo(y, x)$.
    \item Используя полярный угол можно легко нормировать вектор, ведь если $\alpha$ — полярный угол вектора $v(x; y)$, то понятно, что $\v{n} = \v{(\cos \alpha; \sin \alpha)}$ будет искомым вектором. Также можно было бы считать по формуле $\v{n} = \v{\left(\frac{x}{|v|}; \frac{y}{|v|}\right)}$.
    \item Перпендикулярным вектором (одним из) для $v(x, y)$ является $u(-y, x)$, так как их перпендикулярность можно проверить через скалярное произведение.
    \item Ввод и вывод вектора тоже бывает нужен. Для этого определим операторы \lcpp{>>} и \lcpp{<<} с нужными аргументами и C++ будет понимать выражения \lcpp{cin >> v} и \lcpp{cout << v}.
\end{itemize}

Хоть описание выше достаточно большое, но в коде это выглядит достаточно компактно:

\cpp{geo-1}{28}


\subhead{Прямые}
Вот прямые это уже совсем сложный объект хотя бы потому, что для задания одной прямой на плоскости требуется целых три числа :)

Уравнение прямой мы будем рассматривать в общем виде $a x + b y + c = 0$, но иногда для доказательства формул будем переходить к более простым уравнениям двух видов: $y = k x + m$ и $x = t$. Стоит отметить, что уравнений общего вида для одной прямой бесконечно много, ведь можно домножать все коэффициенты на произвольные ненулевые константы. Перейдём же к операциям с прямыми:

\begin{itemize}
    \item Создание прямой по трём коэффициентам или без них (тогда все коэффициенты будут 0) сделаем через конструктор с аргументами по умолчанию.
    \item Создание прямой по двум точкам ($p$ и $q$) это уже достаточно интересно. Если точки лежат на прямой, то они являются корнями её уравнения, а значит нам нужно решить систему:
    \[\begin{multisys}
        \begin{system}
            a p_x + b p_y + c = 0 \\
            a q_x + b q_y + c = 0
        \end{system}
        \iff
        a(p_x - q_x) + b(p_y - q_y) = 0
        \iff
        a(p_x - q_x) = b(q_y - p_y)
    \end{multisys}\]
    Но ведь понятно, что если взять $a = q_y - p_y$ и $b = p_x - q_x$, то это точно будет решением уравнения, ведь левая и правая часть будут иметь одинаковые формулы. Остаётся лишь узнать коэффициент $c$, но это совсем просто: $c = -a p_x - b p_y = -a q_x - b q_y$.
    \item Получение двух точек на прямой. Они нам понадобятся, чтобы, например, находить расстояние от точки до прямой без вывода формул с нуля. Если у нас $b = 0$, то мы имеем уравнение $ax + c = 0$, то есть $x = t = -\frac{c}{a}$, то есть на прямой лежат, например, точки \p{-\frac{c}{a}}{0} и \p{-\frac{c}{a}}{1}. Иначе наша прямая представима в виде $y = kx + m$, где $k = -\frac{a}{b}$ и $m = -\frac{c}{b}$ и на прямой лежат, например, точки \p{0}{-\frac{c}{b}} и \p{1}{-\frac{a + c}{b}}.
    \item Расстояние от точки $p$ до прямой теперь считается совсем просто. Найдём две точки ($q$ и $r$) на прямой и рассмотрим площадь $S$ для $\triangle pqr$. По формулам площадей треугольника имеем: $2S = h_p \cdot qr$ — площадь через высоту и сторону и $2S = pq \cdot pr \cdot |\sin(\widehat{pq;pr})| = |\v{pq} \wedge \v{pr}|$ — площадь через синус угла. Но тогда имеем равенство $h_p = \frac{|\v{pq} \wedge \v{pr}|}{qr}$, по которому мы уже можем легко рассчитать искомое расстояние.
    \item Параллельные прямые, удалённые на заданное расстояние ($r$). Конечно, можно придумать, как находить такие прямые уже имеющимися средствами. Например, найдём точки $p$ и $q$ на прямой, построим вектор $\v{pq}$, найдём $\v{v} \perp \v{pq}$ и сделаем его заданной длины (сначала нормируем, а потом домножим на $r$) и получим $\v{u}$. После этого проведём одну прямую через точки $p + \v{u}$ и $q + \v{u}$, а вторую через $p - \v{u}$ и $q - \v{u}$. Но, оказывается, что такой способ недостаточно точный, и можно придумать точнее.
    
    Вспомним, что если прямая задана в виде $y = kx + m$, то две прямые параллельны, когда их коэффициенты $k$ равны. Это наталкивает на мысль, что если прямые заданы в общем виде и их коэффициенты при переменных ($a$ и $b$) равны, то они будут параллельны. Будем использовать это предположение, а чуть позже мы докажем его, когда будем искать точку пересечения прямых.
    
    Рассмотрим случай параллельных прямых, когда $x_1 = t_1$ и $x_2 = t_2$. Тогда понятно, что $r = |t_1 - t_2|$. С другой стороны коэффициенты $t$ мы можем выразить, как $t = -\frac{c}{a}$, а следовательно $r = \left| -\frac{c_1}{a_1} + \frac{c_2}{a_2} \right| = \left| \frac{c_2 - c_1}{a} \right|$, из нашего предыдущего предположения, что $a$ и $b$ равны.
    
    Теперь более сложный случай, когда $y_1 = k x_1 + m_1$ и $y_2 = k x_2 + m_2$. Если $k = 0$, то такой случай аналогичен предыдущему и мы получим $r = \left| \frac{c_2 - c_1}{b} \right|$, поэтому будем рассматривать случай $k \ne 0$. Пусть $P(0, m_1)$, $Q(0, m_2)$. причём понятно, что $P$ лежит на первой прямой, а $Q$ на второй. Также на первой прямой лежит и точка $R(\frac{m_2 - m_1}{k}; m_2)$, поэтому хочется рассмотреть $\triangle PQR$. Его площадь можно посчитать два раза: $2S = PQ \cdot QR = PR \cdot h_Q$, отсюда имеем расстояние между прямыми $h_Q = \frac{PQ \cdot QR}{PR} = \frac{|m_2 - m_1| \cdot \left| \frac{m_2 - m_1}{k} \right|}{\sqrt{(m_2 - m_1)^2 + \left( \frac{m_2 - m_1}{k} \right)^2}} = \frac{(m_2 - m_1)^2}{|k| \cdot |m_2 - m_1| \sqrt{1 + \frac{1}{k^2}}} = \frac{|m_2 - m_1|}{\sqrt{k^2 + 1}}$. А теперь вспомним, что $k = -\frac{a}{b}$ и $m = -\frac{c}{b}$, а к тому же по нашему предположению коэффициенты $a$ и $b$ равны: $h_Q = \frac{|-\frac{c_2}{b} + \frac{c_1}{b}|} {\sqrt{(\frac{a}{b})^2 + 1}} = \frac{|c_1 - c_2|}{|b| \cdot \frac{1}{|b|} \sqrt{a^2 + b^2}} = \frac{|c_1 - c_2|}{\sqrt{a^2 + b^2}}$.
    
    А теперь заметим, что формула $r = \frac{|c_1 - c_2|}{\sqrt{a^2 + b^2}}$ отвечает всем трём случаям, ведь если один из коэффициентов при переменных ($a$ или $b$) зануляется, то под корнем остаётся только квадрат одного из слагаемых, корень из которого как раз его модуль. Вспоминая исходную задачу. получаем, что у исходных прямых коэффициенты $a$ и $b$ можно сделать такими же, а коэффициент $c$ должен быть $c' = c \pm r \sqrt{a^2 + b^2}$. Как говорилось выше, уравнение прямой в общем виде нe единственно, поэтому на самом деле коэффициенты $a$ и $b$ у параллельных прямых не обязаны быть равны, но параллельные прямые представимы в таком виде.
    
    \item Пересечение двух прямых. Это, пожалуй, вычислительно самое сложное, что мы сегодня рассматриваем, к тому же, результатов у этой операции много: точка (если пересекаются), пустое множество (если параллельны) и прямая (если прямые совпадают). Но, тем не менее, мы приступим к выводу формул. Понятно, как это делать, ведь нужно просто решить систему уравнений для двух прямых и точки пересечения:
    \[\begin{multisys}
        \begin{system}
            a_1 x + b_1 y + c_1 = 0 \\
            a_2 x + b_2 y + c_2 = 0
        \end{system}
        \iff
        \begin{system}
            x = -\frac{b_1 y + c_1}{a_1} \\
            y = -\frac{a_2 x + c_2}{b_2}
        \end{system}
        \iff
        \begin{system}
            x = \frac{b_1}{a_1 b_2}(a_2x + c_2) -\frac{c_1}{a_1} \\
            y = \frac{a_2}{a_1 b_2}(b_1y + c_1) -\frac{c_2}{b_2}
        \end{system}
        \iff
    \end{multisys}\]
    \[\begin{multisys}
        \iff
        \begin{system}
            x\left(1 - \frac{a_2 b_1}{a_1 b_2}\right) = \frac{b_1 c_2}{a_1 b_2} -\frac{c_1}{a_1} \\
            y\left(1 - \frac{a_2 b_1}{a_1 b_2}\right) = \frac{a_2 c_1}{a_1 b_2} -\frac{c_2}{b_2}
        \end{system}
        \iff
        \begin{system}
            x(a_1 b_2 - a_2 b_1) = b_1 c_2 - b_2 c_1 \\
            y(a_1 b_2 - a_2 b_1) = a_2 c_1 - a_1 c_2
        \end{system}
        \iff
        \begin{system}
            x = \frac{b_1 c_2 - b_2 c_1}{a_1 b_2 - a_2 b_1} \\
            y = \frac{a_2 c_1 - a_1 c_2}{a_1 b_2 - a_2 b_1}
        \end{system}
    \end{multisys}\]
    Теперь поймём, когда наша система не имеет решений, то есть, что значит $a_1 b_2 - a_2 b_1 = 0$. Понятно, что первая прямая параллельна вектору $\v{v} = \v{(b_1; -a_1)}$ (потому что если \p{x_0}{y_0} лежит на прямой, то $a x_0 + b y_0 + c = 0$, а значит и $a (x_0 + b) + b (y_0 - a) + c = (a x_0 + b y_0 + c) + (a b - b a) = 0$), аналогично вторая прямая параллельна вектору $\v{u} = \v{(b_2; -a_2)}$. А теперь заметим, что $\v{v} \wedge \v{u} = b_1 \cdot (-a_2) - (-a_1) \cdot b_2 = a_1 b_2 - a_2 b_1 = 0$, следовательно, если система не решилась, то прямые действительно параллельны или совпали.
    
    Хочется разделять эти два случая, поэтому проверим различаются ли все коэффициенты в одинаковое количество раз (если да, то это уравнения одной прямой). Понятно, что это означает необходимость проверить три равенства:
    \[\begin{multisys}
        \begin{system}
            \frac{a_1}{a_2} = \frac{b_1}{b_2} \\
            \frac{b_1}{b_2} = \frac{c_1}{c_2} \\
            \frac{c_1}{c_2} = \frac{a_1}{a_2}
        \end{system}
        \iff
        \begin{system}
            a_1 b_2 = a_2 b_1 \\
            b_1 c_2 = b_2 c_1 \\
            a_2 c_1 = a_1 c_2
        \end{system}
        \iff
        \begin{system}
            a_1 b_2 - a_2 b_1 = 0 \\
            b_1 c_2 - b_2 c_1 = 0 \\
            a_2 c_1 - a_1 c_2 = 0
        \end{system}
    \end{multisys}\]
    Интересно, что первое из этих равенств мы уже проверили, а второе и третье выражения используются в формулах для координаты точки пересечения прямых.
    
    Таким образом, мы умеем разделять случаи взаимного расположения прямых и определять точку их пересечения, а в коде для пересечения удобно использовать оператор \lcpp{^}.
    
    \item Ввод и вывод прямых тоже бывает нужен. Для этого, как и раньше, определим операторы \lcpp{>>} и \lcpp{<<} с нужными аргументами. чтобы использовать \lcpp{cin >> v} и \lcpp{cout << v}.
\end{itemize}

Видно, что прямые действительно достаточно интересный объект, ведь наше текстовое описание достаточно растянулось, но в коде это занимает не так и много места:

\cpp{geo-1}{37}


\subhead{Дополнения}
Понятно, что точки, вектора и прямые — это не все объекты, которые бывают в геометрии. Например, ещё бывают лучи и отрезки, которые представимы как прямые с концами или даже окружности, с которыми мы сегодня не работали. Но хочется считать, что если читателю попадутся лучи и отрезки, то он сможет с ними разобраться, используя описание выше. Окружности же, встречаются реже, но всё же, читатель может разобраться и с ними.

Также следует сделать несколько комментариев по кодированию вычислительной геометрии:

\begin{enumerate}
    \item Старайтесь считать в целых числах, а если этого не избежать, то сравнивайте числа с учётом погрешности, как это сделано в пересечении прямых. Константу $EPS$ нужно выбирать с учётом задачи, например, вполне подходит точность $10^{-9}$, для этого вполне подойдёт вот такой код:
    
    \cpp{geo-1}{1}
    
    \item Бывает, что требуется число $\pi$. Во-первых, его, конечно же, можно запомнить с желаемой точностью, но если запоминать его не хочется, то можно воспользоваться вторым способом — аркфункциями, например арккосинусом:
    
    \cpp{geo-1}{1}
    
    \item В вычислительной геометрии часто численный ответ просят вывести с заданной точностью. Мы проходили раньше, как это делать, но всё же напомним, что для этого в начале программы достаточно написать:
    
    \cpp{geo-1}{1}
    
    После такой строчки числа не будут выводиться в экспоненциальном виде и всегда будут иметь заданное количество знаков (у нас 9) после запятой.
    
    \item Если будете использовать классы, представленные в этом разделе, то в начале нужно будет объявить константы, потом добавить вот эти две строчки:
    
    \cpp{geo-1}{2}
    
    А после этого, можно будет и написать все классы в том порядке, в котором они были представлены (сначала \lcpp{Point}, потом \lcpp{Vector} и в конце \lcpp{Line}). Две строки кода, написанные чуть выше, нужны для сложения точки с вектором (иначе компилятор к моменту использования вектора ни разу не видел объявления этого класса).
    
    \item Кроме базовых операций, о которых говорилось раньше, ещё очень часто требуется определять взаимное расположение трёх точек $a$, $b$ и $c$. А именно, требуется определять, с какой стороны от прямой $ab$ лежит точка $c$ (если смотреть из $a$ на $b$): справа (\term{Clockwise (CW)} — поворот по часовой стрелке), слева (\term{Counter Clockwise (CCW)} — поворот против часовой стрелки) или на $ab$. Делается это с помощью знака псевдоскалярного произведения:
    
    \cpp{geo-1}{12}
\end{enumerate}
