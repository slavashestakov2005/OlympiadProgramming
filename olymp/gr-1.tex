\head{Введение в графы}
Сегодня мы начинаем изучение задач, связанных с графами. Таких задач довольно много (как и алгоритмов на графах), поэтому начнём мы с определений, которые нам потом понадобятся, и способов представления графа в памяти компьютера.


\subhead{Определения}
\begin{itemize}
    \item Основные объекты:
    \begin{enumerate}
        \item Вершина — изображается на картинках, как точка, для удобства нумеруется. Множество всех вершин обозначается $V$ (от \term{vertex}), количество вершин обозначается как $n = |V|$.
        \item Ребро — связь между вершинами, чаще всего двумя (хотя есть и обобщения, например, рёбра соединяют по три вершины); задаётся ребро номерами вершин, которые соединяет. Множество всех рёбер обозначается $E$ (от \term{edge}), количество рёбер обозначается как $m = |E|$.
        \item Граф — совокупность вершин и рёбер: $G(V, E)$.
    \end{enumerate}
    \item Ориентированность:
    \begin{enumerate}
        \item Ориентированный — граф, у которого на рёбрах есть направления (т.е по ребру $e = (u, v)$ из $u$ можно пройти в $v$, а из $v$ в $u$ — нельзя).
        \item Неориентированный — граф, в котором у рёбер нет направлений.
        \item Смешанный — граф, в котором у части рёбер есть направления, а у части — нет.
    \end{enumerate}
    \item Свойства вершин:
    \begin{enumerate}
        \item Степень — количество рёбер вершины.
        \item Входящая и исходящая степени — количество входящих и исходящих рёбер для данной вершины ориентированного графа.
        \item Изолированная вершина — вершина со степенью $0$.
        \item Висячая вершина (или лист) — вершина со степенью $1$.
    \end{enumerate}
    \item Свойства рёбер:
    \begin{enumerate}
        \item Путь — последовательность вершин, в которой каждая следующая вершина (кроме первой) соединена ребром с предыдущей.
        \item Цикл — путь, у которого начальная и конечная вершина совпадают.
        \item Простые путь и цикл — те, в которых рёбра не повторяются.
        \item Петля — ребро, соединяющее вершину саму с собой.
        \item Кратные рёбра — рёбра, соединяющие одинаковые вершины.
    \end{enumerate}
    \item Связность:
    \begin{enumerate}
        \item Компонента связности — подграф, в котором между каждой парой вершин есть путь, без учёта направления рёбер.
        \item Компонента сильной связности — то же, что и обычная компонента связности, но с учётом направления рёбер.
        \item Связный граф — граф с одной компонентой связности.
        \item Несвязный граф — граф, в котором больше одной компоненты связности.
    \end{enumerate}
    \item Другие свойства графа:
    \begin{enumerate}
        \item Взвешенный граф — граф, в котором у рёбер есть числа (веса).
        \item Простой граф — граф без петель и кратных ребер.
        \item Дерево — связный простой граф без циклов.
        \item Лес — граф, в котором каждая компонента является деревом.
        \item Полный граф — простой граф, в котором между всеми парами вершин есть ребро.
        \item $k$-дольный граф — граф, вершины которого можно раскрасить в $k$ цветов так, чтобы не было рёбер между вершинами одного цвета (так называемая правильная раскраска).
        \item Неявный граф — граф, в котором рёбра заданы не в явном виде, а через правило. Например, ходы коня на шахматной доске задают такой граф (в этом случаем мы проводим ребро, по правилу \term{ребро $(u, v)$ есть, если конь может походить из клетки $u$ в клетку $v$}).
        \item Остовное дерево — множество всех вершин и подмножество рёбер, которые образует дерево. Его всегда можно выделить в связном графе, в несвязном есть остовный лес — остовное дерево для каждой компоненты связности.
    \end{enumerate}
\end{itemize}


\subhead{Представление в информатике}
Если мы работаем с неориентированным графом, то каждое ребро $(u, v)$ можно хранить два раза: отдельно для вершины $u$ и отдельно для $v$. Если граф ориентирован, то таких проблем не возникает и каждое ребро можно хранить ровно один раз. В смешанных графах, которые не встречаются в олимпиадном программировании, можно заменить неориентированные рёбра на пару ориентированных.

Если мы работаем со взвешенным графом, то если ребро есть, то мы будем хранить его вес; если же ребра нет, то можно хранить какое-то специальное значение ($0$, $-\infty$ или $+\infty$ в зависимости от задачи), или не хранить ребро вовсе. Если же граф невзвешенный, то можем считать вес каждого ребра равным $1$, что будет просто обозначать наличие ребра.

Теперь мы рассмотрим основные способы представления взвешенного ориентированного графа.

\begin{itemize}
    \item \textbf{Матрица смежности.} Таблица размера $n \times n$, в которой в строке $i$, столбце $j$ мы храним вес ребра из $i$ в $j$. Недостатки — большой расход памяти \O{n^2} и невозможность хранить кратные рёбра.
    \item \textbf{Список смежности.} Для каждой вершины $u$ сохраним список вершин $v$, в которые можно попасть из $u$. Достоинства — легко обойти граф (получить вершины, в которые можно пойти из текущей), малый расход памяти \O{n + m}.
    \item \textbf{Список рёбер.} Просто сохраним информацию о каждом ребре (соединяемые вершины, вес, ориентированность). Достоинство — малый расход памяти \O{m}, из-за чего чаще всего именно так и дают граф во входных данных.
\end{itemize}

Эти способы представления графа легко запомнить, если думать о количестве вершин каждого ребра, которые мы хотим сохранить как дополнительную информацию (0. 1 и 2 соответственно для способов выше).
