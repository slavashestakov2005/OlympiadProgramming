\head{Бинарный поиск}
Теперь мы перейдём к новой концепции, основанной на идеи \term{разделяй и властвуй} (или \term{уменьшай (наполовину) и решай}, по другой классификации). Смысл достаточно прост: пусть у нас есть какая-то большая задача, которую нужно решить, тогда разобьём её на две примерно равных подзадачи, решим их и после как-то объединим решения. Такой алгоритм нам уже встречался, когда мы проходили сортировку слиянием.

Ну что ж, приступим, наконец, к нашему бинарному поиску: пусть у нас есть функция $f$, у которой мы хотим найти корень $x_0$ на участке $[l;r]$, при этом мы знаем, что функция всегда отвечает условию $f(x_1) \leq f(x_2)$ для $l \leq x_1 \leq x_2 \leq r$, а также $f(l) \cdot f(r) < 0$ (аналогично условию, что $l < x < r$). Тогда мы можем поступить следующим образом: возьмём точку $m = \frac{l + r}{2}$ и вычислим $f(m)$. Теперь возможны три варианта:

\begin{enumerate}
    \item Если окажется, что $f(m) = 0$, то корень мы точно нашли и его можно вернуть.
    \item Если же $f(l) \cdot f(m) > 0$, то значит $f(l)$ и $f(m)$ — одного знака, а следовательно $x_0$ точно не может лежать внутри отрезка $[l;m]$, то есть мы можем присвоить $l = m$ не потеряв решений.
    \item Аналогично, при $f(m) \cdot f(r) > 0$, имеем $f(m)$ и $f(r)$ — одного знака, а следовательно $x_0$ точно не может лежать внутри отрезка $[m;r]$, то есть мы можем присвоить $r = m$ не потеряв решений.
\end{enumerate}

Во-первых, заметим, что такой алгоритм пользуется только свойством $f(l) \cdot f(r) < 0$, при этом, после одного шага это свойство сохраняется, а значит, такой алгоритм можно применять до тех пор, пока не будет достигнута нужная точность (на каждом шаге границы отрезка $[l;r]$ сужаются и можно остановиться, когда отрезок станет достаточно маленьким согласно условию задачи). А во-вторых, поймём, что на каждом шаге длина отрезка уменьшается в 2 раза, а значит, наш поиск совершит \O{\log n} шагов, где $n$ — длина диапазона, в котором мы ищем.

И, в-третьих, добавим, что главное от функции — неубывание или невозрастание (далее будем называть это свойство монотонностью, хотя это и не вполне корректно), поэтому функция, отвечающая условию $f(x_1) \geq f(x_2)$ для $l \leq x_1 \leq x_2 \leq r$, тоже бы подошла.


\subhead{Поиск элемента в наборе}
Представим, что перед нами стоит такая задача: дан набор чисел $a$ длины $n$, для которого нам нужно ответить на $q$ запросов вида: есть ли элемент $x$ внутри $a$.

Простое решение, которое можно придумать, будет работать за \O{q \cdot n}: для каждого запроса будем перебирать все элементы и проверять, нашёлся ли запрошенный $x$. Но такой способ явно не оптимальный, ведь можно придумать решение с использованием множеств: сохраним набор $a$ внутрь множества $s$ и операции поиска будем выполнять уже внутри $s$, ведь там они будут работать за \O{\log n}, а значит итоговая сложность алгоритма составит \O{q \log n}. Но такой способ мы использовать не хотим, потому что он требует использование структуры данных, а это влечёт за собой дополнительные накладные расходы. Так что будем придумывать способ, использующий бинарный поиск.

Первое, что нам нужно — монотонная функция $f$, у которой мы будем искать корень. Поскольку требуется монотонность, то вполне логично будет отсортировать весь набор $a$ (для конкретности будем сортировать по неубыванию, хотя по невозрастанию тоже можно). Теперь становится понятно, как нам нужно определить функцию $f$: $f(k) = a_k - x$, где $x$ — искомое значение.

Теперь перейдём к самой реализации алгоритма. Для начала возьмём границы нашего исходного отрезка: $l=-1$ и $r=n$ (наши элементы в $a$ имеют индексы $0,\ 1,\ \ldots,\ n-1$, а такие $l$ и $r$ как бы указывают на фиктивные элементы, которые можно считать $a_{-1} = -\infty$ и $a_n = \infty$).

Также для удобства понимания работы с индексами договоримся, что у нас всегда будут выполняться условия $a_l < x \leq a_r$. Тогда наш алгоритм будет выглядеть следующим образом:

\cpp{com-6}{4}

Понятно, что согласно изначальному условию если ответ и содержится, то только в элементе $a_r$, поэтому если $r = n$ или $a_r > x$, то элемент не нашёлся, а иначе $a_r = x$ и это первый элемент равный $x$, и поэтому такой бинарный поиск называется \term{lower bound}.

Можно было договориться об инварианте $a_l \leq x < a_r$, и тогда если $x$ не нашёлся бы в случаях $l = -1$ и $a_l < x$, а иначе $a_l = x$ было бы последним подходящим элементом. Если бы наш алгоритм возвращал индекс $r$, то такой бинарный поиск назывался бы \term{upper bound}

И понятно, что алгоритм, решающих исходную задачу работал бы за \O{n \log n + q \log n}, где первое слагаемое возникает из-за сортировки, а второе — ответы на запросы.


\subhead{Встроенный бинарный поиск}
Разумеется, в C++ уже есть встроенные функции, которые ищут значения внутри отсортированных контейнеров. Они называются \lcpp{lower_bound}, \lcpp{upper_bound} (откуда же взялись такие названия :)) и \lcpp{binary_search}, использовать их можно так:

\cpp{com-6}{5}

Только перед этим необходимо подключить нужный файл:

\cpp{com-6}{1}

И, кончено же, встроенные алгоритмы бинарного поиска работают за \O{\log n}\footnote{Это справедливо для контейнеров, в которых есть доступ к произвольному элементу; для множеств и отображений нужно использовать методы самих контейнеров, которые мы изучили раньше: \lcpp{s.lower_bound(x);}.}.

\subhead{Бинарный поиск по ответу}
Есть целый набор задач, в которых относительно легко понять, является ли какое-то число ответом, при этом выписать явную формулы для вычисления этого числа достаточно сложно. Именно в таких задачах используется \term{бинарный поиск по ответу}.

Пусть у нас есть $n > 0$ дипломов шириной $w$ и высотой $h$, при этом дипломы нельзя поворачивать и накладывать друг на друга. Нужно найти минимальное целое $x$ такое, что все дипломы войдут на квадратный стенд $x \times x$, не вылезая за его пределы.

Во-первых заметим, что в этой задаче присутствует монотонная функция: если $x$ является ответом, то для всех $x' > x$ стенд $x' \times x'$ также является ответом, а для всех $x' < x$ стенд $x' \times x'$ ответом точно не является. Так что приняв $l=0$ и $r=n \cdot \max(w, h)$ мы сможем сделать бинарный поиск по ответу (стенда $0 \times 0$ точно не хватит, а вот $(n \cdot \max(w, h)) \times (n \cdot \max(w, h))$ — точно хватит).

Теперь остаётся понять, является ли какое-то число $m$ ответом. А сделать это действительно не сложно, ведь на стенд размером $m \times m$ войдёт $\left\lfloor \frac{m}{w} \right\rfloor$ дипломов по горизонтали и $\left\lfloor \frac{m}{h} \right\rfloor$ по вертикали (такие скобочки — это округление вниз к ближайшему целому). А значит итоговая функция, по которой совершается поиск является $f(m) = \left\lfloor \frac{m}{w} \right\rfloor \cdot \left\lfloor \frac{m}{h} \right\rfloor$, и если $f(m) < n$, то стенд слишком маленький, значит нужно сделать $l = m$; а иначе $f(m) \geq n$, а значит стенд достаточного размера, поэтому делаем $r = m$. Итоговый ответ окажется в границе $r$.

Понятно, что сложность такого решения будет \O{\log (n \cdot \max(w, h))}, ведь мы ищем среди $n \cdot \max(w, h)$ потенциальных ответов. Также понятно, что при любом бинарном поиске по ответу нам нужно будет делать \O{\log t} операций поиска, где $t$ — размер диапазона, в котором мы ищем ответ.


\subhead{Вещественный бинарный поиск}
Теперь рассмотрим ещё одну задачу, чтобы понять широту применимости бинарного поиска. Пусть нам нужно вычислить квадратный корень числа $n$ с точностью $10^{-6}$, при этом использовать встроенные функции корня (и возведения в степень) запрещается.

Здесь достаточно очевидно, что бинарный поиск у нас будет по функции $f(m) = m^2$. Если оказалось, что $f(m) = n$, то $m$ — искомое число; если же $f(m) < n$, то такого $m$ ещё мало, поэтому нужно сделать $l = m$; а иначе $f(m) > n$ и делаем $r = m$. 

Единственным отличием от предыдущих задач будет то, что $l$ и $r$ должны быть вещественными числами, ведь используя целые числа мы не сможем достичь нужной точности. Останавливать же поиск мы можем когда разность $r - l$ станет слишком маленькой, например меньше $10^{-7}$. Но поскольку операции с вещественными числами имеют погрешности, то заданная разность может не достигаться и может получиться бесконечный цикл. Поэтому иногда делают бинарный поиск с заданным числом итераций (например 120, потому что $\log_2 10^{36} \approx 120$, а диапазона $10^{36}$ почти всегда хватает, ведь положительные числа в задачах почти всегда лежат в диапазоне $[10^{-18}; 10^{18}]$).

Такой бинарный поиск называется \term{вещественным} (потому что работает с вещественными числами), его сложность, как обычно, пропорциональна \O{\log t}, где $t$ — размер диапазона. Также приятным свойством вещественного бинарного поиска является факт, что после окончания поиска обе границы ($l$ и $r$) оказываются достаточно близки к ответу, поэтому ответом можно считать любое из чисел.


\subhead{Тернарный поиск}
Теперь рассмотрим алгоритм, очень похожий на бинарный поиск — тернарный поиск. Его суть заключается в следующем: пусть у нас есть функция $f(x)$, у которой ровно один минимум $x_0$ (с максимумом аналогично) на отрезке $[l;r]$, разбивающий функцию на две монотонных части $[l;x_0]$ и $[x_0;r]$.

Тогда на каждом шаге мы можем делить текущий отрезок на три части: $m_1 = l + \frac{r - l}{3} = \frac{2l + r}{3}$ и $m_2 = r - \frac{r - l}{3} = \frac{l + 2r}{3}$. И если оказалось, что $f(m_1) \leq f(m_2)$, то на участке $[m_2; r]$ точно нет ответа, поэтому можем присвоить $r = m_2$. Аналогично при $f(m_1) \geq f(m_2)$, на участке $[l; m_1]$ точно нет ответа, поэтому можем присвоить $l = m_1$.

Заканчивать такой поиск нужно, когда отрезок станет достаточно маленьким или после достаточного числа итераций. Сложность тернарного поиск \O{\log t}, где $t$ — длина диапазона (на самом деле понятно, что тернарного поиск основание логарифма $1.5$, а бинарного — $2$, но эта константа в асимптотике опускается, поэтому у поисков одинаковые сложности).

Применять тернарный поиск можно, например, в задачах по геометрии, когда вам не хочется выводить явную формулу. Ещё стоит отметить, что достаточно часто можно обойтись без тернарного поиска, заменив его на бинарный поиск по производной.
